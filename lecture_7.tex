\paragraph{Пример.} $w=P(x,y)=\frac{y}{x^2+y^2},\; w = Q(x,y)=\frac{x}{x^2+y^2},\; \mathbb{E}^2\slash \{(0,0)\}$ --- не является односвязной областью\\

$\pdd{P}{y}(x,y)=\pdd{Q}{x}(x,y); \varGamma=\{(x,y):\,x=\cos t,\, y =\sin t ,\, 0\leqslant t\leqslant 2\pi\}$\\
$\oint_\varGamma Pdx+Qdy = \oint_\varGamma\dfrac{-ydx+xdy}{x^2+y^2}=\int_{0}^{2\pi}dt=2\pi\neq0$
\section{Замена переменных в кратном интеграле.}
\subsection{Преобразование плоских областей}
$\Omega^*\subset \mathbb{E}^2_{(uv)},\; \Omega \subset \mathbb{E}^2_{(x,y)};\;$
\begin{equation} \label{star} F: \overline{\Omega^*}\rightarrow\overline{\Omega};\; F : x=f^1(u,v),\; y=f^2(y,v)
\end{equation}
\begin{enumerate}
	\item $F$ --- взаимно однородное отображение
	\item $F$ --- дважды непрерывна дифференцируема
	\item $\mathcal{J}_F=\frac{D(x,y)}{D(u,v)}\neq0$ в $\overline{\Omega^*}$
\end{enumerate}
$G=F^{-1};\; G : \overline{\Omega} \rightarrow \overline{\Omega^*}\;\quad G: u = g^1(x,y),\; V = g^2(x,y)$
\paragraph{Свойство A.} При отображении $F$ внутренние точки множества  $\overline{\Omega^*}$ переходят во внутренние точки $\overline{\Omega}$.
\paragraph{Свойство Б.} При отображении $F$ гладкая кривая переходит в гладкую кривую.
\begin{proof} \textit{Свойства А:}
	 $w_0=(u_0,v_0)\in \Omega^*\rightarrow z_0=(x_0,y_0)\in?$\\
	 Из \eqref{star} определены $u$ и $v$ как функции переменных $x$ и $y$ в некоторой окрестности т. $z_0,\\ z_0 \in \Omega$  
\end{proof}
\begin{proof} \textit{Свойства Б:}
	$\varGamma^*=\{(u,v): u = \varphi(t), v=\psi(t), \alpha \leqslant t \leqslant \beta  \},\; \varGamma^*\subset \Omega^*,\; \varGamma^*$ --- гладкая кривая, $\varphi,\,\psi$ --- непрерывно дифференцируемы на $[\alpha,\,\beta]$ функции и $\forall\;t\in[\alpha,\,\beta] \hm{\rightarrow}[\varphi'(t)]^2\hm{+}[\psi'(t)]^2\neq0$\\
	$\varGamma = \{(x,y):x = f_1(\varphi(t),\,\psi(t)),\; y = f^2 (\varphi(t),\;\psi(t)),\, \alpha\leqslant t\leqslant\beta\},\; \varGamma\subset \Omega$\\
	$x'(t)=\pdd{f^1}{u} \varphi'(t)+\pdd{f^1}{v}\psi'(t);\quad y'(t)=\pdd{f^2}{u}\phi'(t)+\pdd{f^2}{v}\psi'(t)$.
	Учтем что $\mathcal{J} \neq 0$, предположим что $\exists t_0:\;x'(t_0)=y'(t_{0}=0$ из неравенства якобиана нулю следует $\varphi'(t) = \psi'(t_{0})=0$ --- противоречие

	$\begin{cases}
		x &= f^1(u_0,v),\;y=f^2(u_0,v),\; v \in \mathbb{R}\\
		u^0 &= g^1(x,y) 
	\end{cases}$
		\\
$	\begin{cases}
	x &= f^1(u,v_0),\; y= f^2(u,v_0),\; u \in \mathbb{R}\\
	v^0 &= g^2(x,y)
	\end{cases}$
	
	Кривые каждого из семейств не пересекаются в силу взаимо однозначности $F$, но через каждую точку проходит 2 кривые по одной из каждого семейства.
\end{proof}
\paragraph{Пример.} $x=\rho \cos \varphi,\; y = \rho \sin \varphi,\;\\
(\rho_0,\,\varphi_0)\quad (\rho_0,\,\varphi_0 +2\pi k),\; k\in\mathbb{Z};\; 0 \leqslant \varphi \leqslant 2\pi$\\
$\prod = \{ (\rho, \varphi): 0 < \rho < R,\; 0 < \varphi < 2\pi \};\; F: \prod \longleftrightarrow K\slash K_1,\; \\ K = \{ (x,y): x^2 + y^2 < R^2 \},\; K_1 = \{ (x,y): y=0,\; 0\leqslant x < R \}$\\
$m(K_1)=0.\; \mathcal{J}_F=\rho > 0$ в $\prod$

\paragraph{Пример 2.} $F: x = \rho \cos \varphi \cos \psi,\; y = \rho \sin \varphi \sin \psi$\\
$\prod= \{ (\rho, \varphi, \psi): 0<\rho< R,\; 0 < \varphi < 2\pi,\; -\pi/2<\varphi < \pi/2 \}$\\
$K \slash K_1,\; K = \{ (x,y,z): x^{2} + y^{2} + z^{2} < R^{2} \}$\\
$K_1 = \{ (x,y,z): 0 \leqslant x^{2} + z^{2} < R^2,\; y=0) \},\; m(K_1)=0;\; \mathcal{J}_F = \rho^{2} \cos \psi > 0 $ в $ \prod $

\paragraph{Пример 3.} $F:\; x=\rho \cos \varphi,\; y = \rho \sin \varphi,\; z=z$\\
$\prod = \{ (\rho, \varphi, z)\; 0 < \rho < R,\; 0 < \varphi < 2\pi,\, 0<z<H \}$\\
$K\backslash K_1: K =\{ (x,y,z): x^{2}+y2 < R,\; 0 < z < H   \} $

$K_1= \{ (x,y,z):\; o<x<R,\; y=0,\; 0<z<H \},\; m(K_1)=0;\; \mathcal{J}  = \rho > 0 $ в $\prod$ 
\subsection{Выражение площади в криволинейных координатах.}
$\partial \Omega^* = \{ (u,v):\; u = \varphi(t),\; v = \psi(t),\; \alpha \leqslant t \leqslant \beta \}$\\
$ \partial \Omega = \{ (x,y):\; x=f^1(\varphi(t),\psi(t)),\, y=f^2(\varphi(t),\psi(t)), \alpha \leqslant t \leqslant \beta \} $\\
$\alpha$ и $ \beta $  выбраны таким образом, что $\partial \Omega$ обходится в положительном направлении.\\
$ m(\Omega) = \int_{\partial \Omega} xdy = \int_{\alpha}^{\beta} f^1(\varphi(t),\psi(t))\left[ \pdd{f^2}{u} \varphi'(t)+\pdd{f^2}{v}\psi'(t)\right]dt = \pm \int_{\partial \Omega^*} f^1 \pdd{f^2}{u} du + f^1\pdd{f^2}{v}dv $\\
$P(u,v) = f^1 \Pdd{f^2}{u},\; Q(u,v) = f^1\Pdd{f^2}{v}\;$

$\Pdd{P}{v}=\Pdd{f^1}{v}\Pdd{f^2}{u} + f^1 \Pdd{^2f^2}{v\partial u},\; \Pdd{Q}{u} = \Pdd{f^1}{u}\Pdd{f^2}{v} + f^1 \Pdd{^2f^2}{u \partial v}$

$ \Pdd{Q}{u} - \Pdd{P}{v} = \Pdd{f_1}{u}\Pdd{f^2}{v} - \Pdd{f^1}{v}\Pdd{f^2}{u} = \mathcal{J}_F(u,v) $

$ m(\Omega) = \iint_{\Omega^*} \abs{\mathcal(J)_F(u,v)} dudv$

\begin{sentence}
	Если $\mathcal{J}_F(u,v) > 0$, то положительному обходу $\partial \Omega^*$ соответствует положительному обходу $\partial \Omega$. Если $\mathcal{J}_F(u,v)<0$, то положительный обход $\partial \Omega^*$ соответствует отрицательный обход $\partial \Omega$ 
\end{sentence}
\subsection{Геометрический смысл модуля якобиана.}
$ [\alpha, \beta] \subset [\alpha_2, \beta_2] \subset \ldots \subset [\alpha_k,\beta_k] \subset \ldots;\; \delta_k = \beta_k - \alpha_k \rightarrow 0 $ при $ k \rightarrow \infty $

т. Кантора $\exists! x_0: x_0 \in [\alpha_k, \beta_k] \forall k $ 

$ y = f(x) $ --- строго монотонна на $ [\alpha, \beta] $, непрерывна на $ [\alpha, \beta] $ и дифференцируема на $ (\alpha, \beta) $, тогда:\\
$ \forall k \exists x_k \in (\alpha_k,\beta_k): f(\beta_k) - f(\alpha_k) = f'(x_k)\cdot\delta_k ;\; B_k=f(\beta_k),\,A_k=f(\alpha_k),\; \Delta k$ для отрезка $ [A_k, B_k] $ или $[B_k, A_k] $\\
$\abs{f'(x_k)} = \frac{\Delta k}{\delta k},\; k \rightarrow \infty,\; x_k \rightarrow x_0;\; \abs{f'(x_0)} = \lim\limits_{k\rightarrow\infty} \frac{\Delta K}{\delta K}$\\
$ \overline{Q^*_k} \subset \Omega^*;\; \overline{Q^*_1} \subset \overline{Q^*_2}\subset \ldots \subset \overline{Q^*_k} \subset \ldots$ причем $ m(\overline{Q^*_k}) \rightarrow 0( при\,k\,\rightarrow \infty) \Rightarrow \exists$ единственная точка $\omega_0=(u_0,v_0): \omega_0 \in \overline{Q^*_k}\;\forall k$\\
$\exists \omega_k \in \overline{\Omega^*_k}: m(\overline{Q^*_k}) = \abs{\mathcal{J}_F ( \omega_k)} m(\overline{Q^*_k}),\; \overline{Q^*_k} = F(\overline{Q^*_k}),\; \omega_k \rightarrow \omega_0\,$ при $ k\rightarrow\infty $ значит:\\
$ \abs{\mathcal{J}(\omega_0)} = \lim\limits_{k \rightarrow \infty} \dfrac{m(\overline{\Omega})}{m(\overline{\Omega^*_k})}$ --- коэффициент растяжения точки $ \omega_0 $ плоскости переменных $ u,v $ при заданном отображении $F$ в $x,y$.